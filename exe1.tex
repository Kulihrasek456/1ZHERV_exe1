%%%%%%%%%%%%%%%%%%%%%%%%%%%%%%%%%%%%%%%%%%%%%%%%%%%%%%%%%%%%%%%%%%%%%%%%%%%%%%%%
% Author : David Radek, Tomas Polasek (template)
% Description : First exercise in the Introduction to Game Development course.
%   It deals with an analysis of a selected title from the point of its genre, 
%   style, and mechanics.
%%%%%%%%%%%%%%%%%%%%%%%%%%%%%%%%%%%%%%%%%%%%%%%%%%%%%%%%%%%%%%%%%%%%%%%%%%%%%%%%

\documentclass[a4paper,10pt,english]{article}

\usepackage[left=2.50cm,right=2.50cm,top=1.50cm,bottom=2.50cm]{geometry}
\usepackage[utf8]{inputenc}
\usepackage{hyperref}
\hypersetup{colorlinks=true, urlcolor=blue}

\newcommand{\ph}[1]{\textit{[#1]}}

\title{%
Analysis of Mechanics%
}
\author{%
David Radek (xRADEKD00)%
}
\date{}

\begin{document}

\maketitle
\thispagestyle{empty}

{%
\large

\begin{itemize}

\item[] \textbf{Title:} \ph{Unrailed}

\item[] \textbf{Released:} \ph{2019}

\item[] \textbf{Author:} \ph{Indoor Astronaut,  Daedalic Entertainment}

\item[] \textbf{Primary Genre:} \ph{ Strategy, Multiplayer  }

\item[] \textbf{Secondary Genre:} \ph{ Real-Time, Cooperative  }

\item[] \textbf{Style:} \ph{Cartoon, low-poly, isometric, simplistic}

\end{itemize}

}

\section*{\centering Analysis}

\subsection*{How the Primary and Secondary Genres are Reflected in the Gameplay}

Unrailed is a cooperative multiplayer game where players work together in real-time to build and maintain a railway track, ensuring that the train does not derail. Its primary genre, strategy, is deeply embedded in the gameplay as players must plan routes, allocate resources, and manage tasks efficiently. Cooperation and communication are essential, reflecting the secondary genres of real-time and cooperative gameplay. These genres combine seamlessly, creating a dynamic and engaging experience. The real-time aspect adds urgency and makes the game more chaotic and prone to create fun and engaging experiences.

The secondary genres strongly enhance the primary genre. The real-time element amplifies the strategic challenge by forcing players to make decisions quickly and under pressure. For example, gathering materials such as wood and iron or clearing obstacles requires coordination to avoid delays. The cooperative gameplay further supports this dynamic, as players must assign roles such as track laying, resource gathering or gathering water for the train. 

The cartoonish, low-poly style of Unrailed complements its fast-paced and accessible gameplay. The simplistic visuals ensure that the action remains clear and uncluttered, even during chaotic moments. The isometric perspective provides a broad view of the environment, enabling players to assess the terrain and plan their actions effectively.


\subsection*{Instructions}

In this assignment, you are tasked with the analysis of a selected game-related title. The title may be a game, video game, serious game, or even serious application using game development tools. Your goal is to analyze the title from the point of its genres and style. As a part of this template, there are some placeholders and hints \ph{like this one}, which you should read and potentially replace with your own text.

Finally, move to the \ph{free-form text} part of the analysis in the form of short prose. Images should be used sparingly and best avoided them entirely. You should primarily focus on: 
\begin{enumerate}
    \item How are the primary and secondary genres reflected in the gameplay?
    \item How do the primary and secondary genre interact? Do the secondary genres support the primary genre? Do they enhance the game, or are they detrimental?
    \item Does the style support the gameplay? Why was it chosen?
\end{enumerate}

\subsection*{Content}

After selecting the \ph{title}, you should first find out when it was \ph{first released} and who \ph{created it}. Be sure to consider the actual information if you choose a re-iteration or ``enhanced edition.'' 

Next, look at the game (or, even better, play it!) and determine the \ph{primary genre}. This genre should be the one supporting the core gameplay. You can use any genre taxonomy (not just the one from the lectures), but keep it unambiguous. A Game can have multiple modes of play -- e.g., Minecraft with creative and survival modes -- in which case you can choose any number of them, but be sure to emphasize your choice in the analysis.

After these steps, look at the \ph{secondary genres} and select one or more of them. Using Survival Minecraft as an example, we have a role-playing sandbox (primary) combined with the casual building and a hint of roguelike with the hardcore mode (secondary). Finally, determine the game's \ph{style} -- a combination of visual, aural, tactile, etc. For example, Minecraft can be considered a retro or cartoon-styled game.

\newpage
\section{The thesis itself}
Unrailed is a cooperative multiplayer game where players work together in real-time to build and maintain a railway track, ensuring that the train does not run out of track before reaching the station. Tracks have to be built at one of the trains wagons and require one wood, one stone from the cargo wagons. The train needs some attention too as it overheats quickly and needs to be cooled with water buckets (which need to be filled at lakes around the map). 
There are many wagon types supporting an endless amount of game strategies while keeping it simple enough for most of potential players. Every one of these wagons can be upgraded by 4 levels and the upgrades are positive only, which simplifies the leveling system and makes it predictable.

\subsection*{How the Primary and Secondary Genres are Reflected in the Gameplay}
Its primary genre, strategy, is reflected in the gameplay as players must plan routes, allocate resources, and manage tasks efficiently.

The secondary genres of real-time and cooperative gameplay combine seamlessly with the primary genre, creating a dynamic and engaging experience. The real-time aspect adds urgency, while the cooperative nature requires teamwork between players and shared responsibility. Though one can play alone if they don't have enough controlers (or friends) because the game includes a single-player mode where there are two characters, one is controlled by the player and the other one can be given simple orders. The player can switch between controling one or the other at any time thus allowing them to multitask very efficiently.

\subsection*{Interaction Between Primary and Secondary Genres}

The secondary genres strongly enhance the primary genre. The real-time element amplifies the strategic challenge by forcing players to make decisions quickly and under pressure. For example, gathering materials such as wood and iron (which are required for the construction of new track pieces) or clearing obstacles requires coordination to avoid delays and softlocks (because players can get blocked by a train and many of good runs end this way). The cooperative gameplay further supports this, as players must assign roles such as track laying, resource gathering, or water bucket handling to succeed. Together, these secondary genres heighten the tension and reward effective collaboration, making the game both challenging and enjoyable.

\subsection*{Does the Style Support the Gameplay? Why Was it Chosen?}

The cartoonish, low-poly style of Unrailed complements its fast-paced and accessible gameplay. The simplistic visuals ensure that the action remains clear and uncluttered, even during chaotic moments and makes it easier to point places of interest to other players. The isometric perspective provides a broad view of the environment, enabling players to see the terrain early enough to plan their prefered routes. The playful aesthetic reduces the stress of the time-sensitive mechanics, making the game approachable for a wide audience.

\subsection{Other gamemodes}
Even though Unrailed is a very simple game, it supports multiple gamemodes that change the reason of why you want to get to the station. 

\textbf{endless} -
Pretty much the standard mode. You start in the grasslands and are tasked with a random sidequest which award you bolts if you complete them. You can also find hidden bolts around the map which can be used to upgrade the train at every station. 

\textbf{quick} -
You start on a random level with a base train and your goal is to get to the next station. No bolts or sidequests are available and the game ends when you get to the station.

\textbf{versus} -
This mode allows you and your friends to split up into two teams. Every team has their own train and map and your goal is to reach the station faster that the other team (or get further if they fail).
Both maps are exactly the same so it's fair for both teams.

\textbf{time} -
In time mode, you have 20 minutes to get as far as you can. There is no upgrading sections, instead you get an upgrade box (which you can use on any part of the train to upgrade it) at every station.

\textbf{sandbox} -
Sandbox starts in the upgrading section but you have unlimited bolts and all trains and wagons are available. Once happy with your train, you cen start the game and try it out.


\newpage
\subsection*{Formatting \& Submission}

Your submission must follow a similar \textbf{structure} as this template. You can either use the provided \LaTeX\ template or roughly replicate it in some other text processing software. The format of the analysis section is left up to you -- you can include sub-sections or write one long text. However, your whole document \textbf{must fit} on exactly one page of \textbf{A4}. The only accepted document format is \textbf{pdf}. Finally, you can submit the pdf by following the submission guidelines detailed on the \href{http://cphoto.fit.vutbr.cz/ludo/courses/izhv/exercises/sub/}{course's website}.

\end{document}
